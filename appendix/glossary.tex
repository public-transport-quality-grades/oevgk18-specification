% Append the following to the XeLaTex command: |makeglossaries %

\newglossaryentry{ÖV-Güteklassen}{
    name={ÖV-Güteklassen},
    description={Geben Auskunft darüber, wie gut ein Standort mit dem öffentlichen Verkehr erschlossen ist.}
}

\newglossaryentry{OeVGK93}{
    name={OeVGK93},
    description={Steht kurz für ÖV-Güteklassen 93 und bezeichnet die Definition der ÖV-Güteklassen, welche im Jahre 1993 mit der Schweizer Norm 640 290 verabschiedet wurde.}
}

\newglossaryentry{OeVGK18}{
    name={OeVGK18},
    description={Steht kurz für ÖV-Güteklassen 2018 und bezeichnet die neue Spezifikation, welche im Zuge dieser Arbeit erarbeitet wird.}
}

\newglossaryentry{Leistungskilometer}{
    name={Leistungskilometer},
    description={Der Leistungskilometer berücksichtigt die Horizontaltdistanz sowie die Werte, die sich aus Steigung und starker Gefälle errechnen lassen und ist ein Mass zur Abschätzung des Zeit- und Energieaufwands.}
}

\newglossaryentry{Haltestelle}{
    name={Haltestelle},
    plural={Haltestellen},
    description={Wird, wenn nicht anders erwähnt, als eine ÖV-Haltestelle verstanden, an der ein öffentliches Verkehrsmittel hält. Dies kann z.B. ein Bahnhof, eine Bus- oder Tramhaltestelle oder auch eine Berg- oder Talstation einer Seilbahn sein.}
}