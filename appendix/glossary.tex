% Append the following to the XeLaTex command: |makeglossaries %

\newglossaryentry{ÖV-Güteklassen}{
    name={ÖV-Güteklassen},
    description={ÖV-Güteklassen geben Auskunft darüber, wie gut ein Standort mit dem öffentlichen Verkehr erschlossen ist.}
}

\newglossaryentry{OeVGK93}{
    name={OeVGK93},
    description={Steht kurz für ÖV-Güteklassen 93 und bezeichnet die Definition der ÖV-Güteklassen, welche im Jahre 1993 mit der Schweizer Norm 640 290 verabschiedet wurde.}
}

\newglossaryentry{OeVGK18}{
    name={OeVGK18},
    description={Steht kurz für ÖV-Güteklassen 2018 und bezeichnet die neue Spezifikation, welche im Zuge dieser Arbeit erarbeitet wird.}
}

\newglossaryentry{Terrainmodell}{
    name={Terrainmodell},
    description={Ein digitales Terrainmodell (DTM) beschreibt die Geländeform ohne Bewuchs und Bebauung.}
}

\newglossaryentry{Leistungskilometer}{
    name={Leistungskilometer},
    description={Der Leistungskilometer berücksichtigt die Horizontaltdistanz sowie die Werte, die sich aus Steigung und starker Gefälle errechnen lassen und ist ein Mass zur Abschätzung des Zeit- und Energieaufwands.}
}

\newglossaryentry{Isochrone}{
    name={Isochrone},
    plural={Isochronen},
    description={Eine Darstellung der Erreichbarkeit auf einer Karte von einem Standort aus. Definiert wird dies als geschlossene Linie um einen Standort, die alle Punkte miteinander verbindet, die mit gleicher Laufzeit von diesem Standort entfernt sind.}
}

%TODO GeoJSON Feature