\chapter{Einführung}
\label{Einführung}

\section{Grundlagen und Begriffe}
\label{Einführung:Grundlagen und Begriffe}

Nachfolgend werden Grundlagen und Begriffe eingeführt, welche für das grundlegende Verständnis des Dokuments relevant sind. Für zusätzliche Definitionen kann das Glossar zur Hand gezogen werden.

\subsection{ÖV-Güteklasse}
\label{Grundlagen und Begriffe:ÖV-Güteklasse}

\subsubsection{Definition}
% same description is used in the corresponding glossary entry
"`\acs{ÖV}-Güteklassen geben Auskunft darüber, wie gut ein Standort mit dem öffentlichen Verkehr erschlossen ist.
Dies ist wichtig, wenn es darum geht, den öffentlichen Verkehr zu optimieren, Siedlungsverdichtung nach Innen an geeigneten Lagen voranzutreiben oder Standortentscheide für publikumsintensive Anlagen so zu treffen, dass sie möglichst wenig zusätzlichen Autoverkehr verursachen."'~\cite{oev-guteklasse-gr-defintion}

\subsubsection{OeVGK93}
% same description is used in the corresponding glossary entry
OeVGK93 steht kurz für \acs{ÖV}-Güteklassen 93 und bezeichnet die Definition der \acs{ÖV}-Güteklassen, welche im Jahre 1993 mit der \acs{SN} 640 290~\cite{sn640290} verabschiedet wurde.

\subsubsection{OeVGK18}
% same description is used in the corresponding glossary entry
OeVGK steht kurz für \acs{ÖV}-Güteklassen 2018 und bezeichnet die neue Spezifikation der \acs{ÖV}-Güteklassen, welche in diesem Dokument beschrieben wird.

\cleardoublepage
\section{Dank}
\label{Resultate:Dank}

Wir möchten folgenden Personen für ihre Unterstützung und Mitwirkung beim Erarbeiten dieser Spezifikation danken:

\textbf{Prof. Stefan Keller, IFS Institut für Software}

\textbf{Prof. Claudio Büchel, IRAP Institut für Raumentwicklung}